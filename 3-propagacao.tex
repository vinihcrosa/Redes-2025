\section{Vetores de Propagação}

Com base na literatura analisada, ransomware se propaga principalmente por cinco mecanismos:

\subsection{Phishing e Engenharia Social}

É o vetor mais comum (≈ 60–70\% dos ataques).
\begin{itemize}
  \item E-mails com anexos maliciosos.
  \item Links para websites comprometidos.
  \item Documentos do Office com macros.
\end{itemize}

\subsection{Exploração de Vulnerabilidades (CVEs)}

Segundo Saccone et al., a exploração de falhas públicas é o principal método de acesso inicial em ataques modernos.

As CVEs mais exploradas envolvem:
\begin{itemize}
  \item Vulnerabilidades em serviços de rede expostos.
  \item Falhas em softwares amplamente usados.
  \item Protocolos de comunicação inseguros.
\end{itemize}

Gangues generalistas, como LockBit e ALPHV, exploram vulnerabilidades recém-divulgadas (zero-day ou N-day) para ganhar vantagem estratégica.

\subsection{Ataques a RDP e Credenciais Comprometidas}

\begin{itemize}
  \item Quebra de senhas fracas.
  \item Uso de credenciais vazadas na dark web.
	\item Abuso de protocolos de acesso remoto.
\end{itemize}

\subsection{Cadeia de Suprimentos}

Ataques via fornecedores ou atualizações comprometidas.

\subsection{Propagação Interna Automatizada}
Após o acesso inicial, o ransomware muitas vezes se replica pela rede usando:
  
	\begin{itemize}
	\item Mimikatz para extração de credenciais,
	\item Execução de comandos remotos,
	\item Exploração de SMB e Active Directory mal configurado.
	\end{itemize}