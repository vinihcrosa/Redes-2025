\section{Vetores de Propagação}

Os estudos recentes convergem para um conjunto recorrente de vetores de propagação. As pesquisas de Cen et al. \cite{CEN2024110138}, Alzahrani et al. \cite{Alzahrani_2025} e Saccone et al. \cite{SACCONE2025104264} mostram que os operadores combinam ataques de phishing, exploração de CVEs, abuso de credenciais e efeitos em cadeia para maximizar a taxa de entrada e cobertura na rede vítima.

\subsection{Phishing e Engenharia Social}

Campanhas de \textit{malspam} continuam sendo a forma mais frequente de entrega inicial, aproveitando anexos com macros ou links para \textit{payloads} hospedados externamente \cite{CEN2024110138}. O trabalho de Shaikh et al. evidencia que a expansão do trabalho remoto aumentou a superfície de ataque para e-mails maliciosos e mensagens de spear-phishing \cite{shaikh2024}. Saccone et al. mostram que gangues "especialistas" dependem mais de phishing e reutilização de credenciais do que grupos generalistas \cite{SACCONE2025104264}.

\subsection{Exploração de Vulnerabilidades (CVEs) e Serviços Expostos}

A base de 16 mil incidentes analisada por Saccone et al. revela que a técnica T1190 (exploração de serviços expostos) é um dos pontos mais usados no \textit{kill chain}, com destaque para falhas de alta severidade \cite{SACCONE2025104264}. O modelo de Ransomware-as-a-Service descrito por Alzahrani et al. indica que afiliados exploram CVEs antigas em sistemas desatualizados, mantendo baixo custo de intrusão \cite{Alzahrani_2025}. Na fase inicial, Cen et al. listam exploração remota (RCE) e kits de \textit{exploit} como vetores típicos \cite{CEN2024110138}, reforçando a necessidade de gestão de patches e redução de superfície exposta.

\subsection{Ataques a RDP, VPN e Credenciais Comprometidas}

Segundo Cen et al., serviços de área remota com senhas fracas ou reaproveitadas são frequentemente violados por \textit{brute force} para entrega do \textit{payload} \cite{CEN2024110138}. Saccone et al. apontam que grupos especializados em setores específicos privilegiam credenciais vazadas ou compradas para evitar ruído de exploração \cite{SACCONE2025104264}. Esses vetores aparecem no estágio P1 da cadeia de ataque de Cen et al., antes mesmo de qualquer persistência ou movimento lateral \cite{CEN2024110138}.

\subsection{Cadeia de Suprimentos e Atualizações Comprometidas}

Cen et al. destacam que atualizações de software e dependências confiáveis podem ser adulteradas para entregar o ransomware, explorando a relação de confiança entre fornecedor e cliente \cite{CEN2024110138}. Esse vetor dilui a origem do ataque e dificulta a detecção precoce, pois o tráfego aparenta ser legítimo e assinado.

\subsection{Propagação Interna Automatizada}

Após o acesso inicial, a movimentação lateral tende a ser rápida e automatizada. Cen et al. descrevem o uso de protocolos de compartilhamento de arquivos e ferramentas internas para ampliar o alcance do \textit{payload} \cite{CEN2024110138}, enquanto Saccone et al. mapeiam a sequência TTP que vai de escalada de privilégios à criptografia (T1486) \cite{SACCONE2025104264}. Estudos focados em tráfego, como Akibis et al. \cite{alibis2024measuring}, medem a velocidade de varredura e variação de pacotes durante esse movimento lateral, fornecendo indicadores para detecção antes da cifragem. Rollere et al. mostram que gráficos de correlação temporal destacam essa aceleração do comportamento e ajudam a sinalizar hosts recém-comprometidos \cite{rollere2025algorithmicsegmentationbehavioralprofiling}.

\subsection{Indicadores Operacionais e Mitigação em Rede}

Vários trabalhos sugerem sinais práticos para interromper a propagação. Anomalias em tráfego leste-oeste, especialmente picos de conexões SMB ou RDP entre hosts que não se comunicam normalmente, podem indicar varredura automática \cite{alibis2024measuring,CEN2024110138}. A correlação de eventos de autenticação falha com tentativas de execução remota (PsExec/WMIC) ajuda a separar ruído de comportamento malicioso \cite{SACCONE2025104264}. Em paralelo, o monitoramento de chamadas iniciais a bibliotecas criptográficas, combinado com padrões de I/O, gera alertas de pré-criptografia (\textit{PERD}) antes do impacto \cite{Begovic_2023,shaikh2024}.

Em termos de mitigação, segmentação de rede e aplicação de \textit{deny-by-default} em serviços de administração reduzem caminhos de movimento lateral \cite{enisa_2023}. \textit{Rate limiting} para logins remotos e uso de MFA/rotação de credenciais mitigam \textit{brute force} e abuso de contas válidas. A priorização de patches para CVEs exploradas no \textit{blueprint} (RCEs em perímetro) reduz a superfície inicial \cite{SACCONE2025104264}. Por fim, \textit{honey tokens} e compartilhamentos-isca em segmentos internos criam pontos de detecção de baixo custo para identificar varredas automatizadas antes que atinjam ativos críticos \cite{rollere2025algorithmicsegmentationbehavioralprofiling}.
