\section{Introdução}

O ransomware se consolidou como uma das principais ameaças cibernéticas globais, causando interrupções em setores críticos e perdas financeiras expressivas. Relatórios recentes apontam crescimento contínuo em volume e sofisticação, impulsionado pelo uso de criptografia robusta, técnicas de evasão e ofensivas coordenadas contra serviços expostos \cite{enisa_2023,Begovic_2023}. O impacto econômico é amplificado por modelos de dupla e tripla extorsão, que combinam cifragem, exfiltração e chantagem pública.

Esse cenário é sustentado por um ecossistema industrializado de \textit{Ransomware-as-a-Service} (RaaS), que padroniza ferramentas e \textit{playbooks} e permite que afiliados adotem táticas conforme o alvo \cite{SACCONE2025104264,Alzahrani_2025}. Estudos recentes descrevem um \textit{blueprint} recorrente: exploração de CVEs críticas ou credenciais válidas, movimento lateral acelerado, sabotagem de recuperação e extorsão estruturada \cite{SACCONE2025104264}. Em paralelo, há aumento de campanhas oportunistas sustentadas por \textit{phishing}, kits de exploit e cadeias de suprimento comprometidas \cite{enisa_2023}.

A pesquisa também avança em mensurar padrões de propagação na rede e em fortalecer a detecção pré-encriptação (\textit{Pre-Encryption Ransomware Detection} — PERD), combinando análise comportamental e aprendizado de máquina \cite{CEN2024110138,alibis2024measuring,shaikh2024,rollere2025algorithmicsegmentationbehavioralprofiling}. Esses trabalhos evidenciam que a proteção eficaz depende de identificar sinais iniciais, antes que a cifragem torne a recuperação inviável \cite{Begovic_2023}.

Este artigo resume evidências da literatura recente e aplica-as ao estudo da propagação e do modus operandi de ransomware. Mapeamos a evolução histórica, descrevemos vetores de entrada e movimentação lateral, detalhamos a cadeia de ataque típica e analisamos estratégias de gangues e tendências de detecção precoce, com o objetivo de orientar práticas de mitigação baseadas em dados.
