\section{Introdução}

Ransomware passou de códigos experimentais em disquete (1989) para um ecossistema industrializado, alimentado por \textit{Ransomware-as-a-Service} (RaaS) e cadeias de suprimento comprometidas. Pesquisas recentes mostram que incidentes atingem desde indivíduos até infraestruturas críticas, combinando exploração de serviços expostos, phishing, movimento lateral automatizado e criptografia forte que inviabiliza recuperação sem chave \cite{CEN2024110138,SACCONE2025104264,Begovic_2023}.

O impacto é econômico e sistêmico: bilhões em perdas anuais, interrupções em saúde, energia e governo, e pressão reputacional ampliada por dupla/tripla extorsão \cite{enisa_2023}. O modelo RaaS democratiza o acesso ao crime, permitindo que afiliados escolham TTPs conforme alvo e pressa operacional \cite{Alzahrani_2025,SACCONE2025104264}.

Neste trabalho investigamos vetores de propagação, modus operandi e estratégias de detecção precoce, conectando evidências empíricas e pesquisas acadêmicas para fundamentar recomendações de mitigação.
