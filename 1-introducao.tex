\section{Introdução}

Ransomware evoluiu de um malware rudimentar distribuído por disquetes em 1989 para um ecossistema industrializado e altamente sofisticado, capaz de paralisar governos, empresas e infraestruturas críticas. Ataques recentes exploram falhas em sistemas expostos à internet, realizam movimento lateral altamente automatizado e utilizam criptografia de nível militar para tornar a recuperação inviável sem o pagamento de resgate.

Além dos danos financeiros — que ultrapassam bilhões de dólares anuais — os ataques afetam serviços essenciais, como hospitais, transporte, energia e cadeias de suprimento. Segundo \cite{SACCONE2025104264}, o crescimento das operações de Ransomware-as-a-Service democratizou o acesso ao crime digital, permitindo que indivíduos sem conhecimento técnico utilizem plataformas completas de ataque.

Diante desse cenário, compreender como ransomware se propaga e opera é fundamental para fortalecer medidas de defesa, prevenção e detecção precoce.