
\documentclass{article}
\usepackage[utf8]{inputenc}
\usepackage{geometry}
\geometry{
    a4paper,
    left=3cm,
    right=2cm,
    top=3cm,
    bottom=2cm
}

\begin{document}

\section{Detecção Precoce de Ransomware (PERD)}

A Detecção de Ransomware Pré-Criptografia (\textit{Pre-Encryption Ransomware Detection} — PERD) tem como objetivo identificar ataques antes da fase de criptografia, mitigando perdas irreversíveis de dados e interrupções operacionais. Diante da crescente sofisticação das ameaças, essa abordagem tornou-se um componente estratégico essencial da segurança cibernética.

\section{Pontos de Concordância Entre os Estudos}

Os trabalhos analisados demonstram consenso quanto à gravidade do \textit{ransomware}, à necessidade de detecção proativa e às limitações das defesas tradicionais. Os principais pontos convergentes são:

\subsection{Evolução do Ransomware como Ameaça Crítica}

Os estudos reconhecem o \textit{ransomware} como uma das maiores ameaças da atualidade, com crescimento exponencial de ataques em escala global. Esse cenário é impulsionado pelo uso de algoritmos criptográficos eficientes, como AES e Salsa20, aliados a técnicas avançadas de ofuscação e \textit{packing}, que reduzem a eficácia de mecanismos baseados em assinaturas.

\subsection{Centralidade da Detecção Pré-Criptografia}

Há consenso de que a interrupção do ataque antes da fase de criptografia é crucial. Após a criptografia, especialmente com esquemas assimétricos, a recuperação dos dados torna-se inviável na prática. Assim, a PERD é vista como essencial para minimizar impactos financeiros e operacionais.

\subsection{Machine Learning como Abordagem Dominante}

Os artigos convergem ao indicar \textit{Machine Learning} e \textit{Deep Learning} como métodos mais eficazes para detecção moderna. Esses modelos se destacam pela capacidade de capturar padrões comportamentais complexos e de se adaptar a variantes \textit{zero-day}, especialmente em ambientes com grande volume de dados.

\subsection{Impacto do Ransomware-as-a-Service (RaaS)}

Os estudos apontam o modelo \textit{Ransomware-as-a-Service} como um fator central para a expansão do ecossistema de ataques, ao reduzir a barreira técnica de entrada. Isso contribui para o aumento da frequência, escala e diversidade dos incidentes, ampliando a superfície de risco organizacional.

\subsection{Limitações dos Conjuntos de Dados}

Há forte consenso quanto à ausência de \textit{datasets} padronizados. A dependência de bases próprias, muitas vezes desatualizadas, compromete a reprodutibilidade e a comparabilidade entre estudos, sendo apontada como um dos principais gargalos da área.

\section{Pontos de Divergência Entre os Estudos}

Apesar do consenso conceitual, os artigos divergem quanto à operacionalização da PERD e às estratégias técnicas adotadas.

\subsection{Definição da Fase Pré-Criptografia}

As abordagens variam entre:
\begin{itemize}
    \item uso de janelas temporais fixas após a execução;
    \item detecção da primeira chamada a APIs criptográficas;
    \item correlação entre chamadas de API e eventos de I/O (IRP).
\end{itemize}

As duas primeiras são simples, porém menos precisas, enquanto a última é mais acurada, mas depende fortemente de sincronização temporal.

\subsection{Abordagens de Análise}

A análise estática é descrita como rápida, porém vulnerável a ofuscação. Já a análise dinâmica é considerada mais eficaz contra técnicas de evasão, mas apresenta maior custo computacional e é suscetível a técnicas de \textit{anti-debugging}.

\subsection{Foco de Plataforma}

Os estudos concentram-se majoritariamente em ambientes Windows, devido à maior incidência de ataques. Há menor atenção a plataformas móveis, com predominância de pesquisas voltadas ao Android em comparação ao iOS.



\end{document}
