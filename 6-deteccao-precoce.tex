\section{Detecção Precoce de Ransomware (PERD)}

A detecção pré-criptografia busca interromper o ataque antes que a cifragem torne a recuperação impraticável. Pesquisas recentes reforçam que a fase pré-impacto concentra sinais úteis, especialmente quando combinam análise comportamental e técnicas de aprendizado de máquina \cite{CEN2024110138,shaikh2024,rollere2025algorithmicsegmentationbehavioralprofiling}. O aumento do volume e da velocidade das campanhas — impulsionado por RaaS — torna a PERD peça central de resiliência \cite{Alzahrani_2025}.

\subsection{Convergências na Literatura}
\textbf{Ameaça crítica.} O ransomware é tratado como uma das principais ameaças atuais, sustentado por criptografia robusta e ofuscação que limitam assinaturas tradicionais \cite{Begovic_2023,CEN2024110138}.\\
\textbf{Centralidade da fase pré-criptografia.} Há consenso de que, após a cifragem (especialmente assimétrica), a recuperação é inviável; detectar antes do impacto é essencial para reduzir perdas \cite{Begovic_2023,shaikh2024}.\\
\textbf{Predominância de ML/DL.} Modelos de \textit{machine learning} e \textit{deep learning} dominam as propostas, pela capacidade de capturar padrões complexos e adaptar-se a variantes \textit{zero-day} \cite{shaikh2024,rollere2025algorithmicsegmentationbehavioralprofiling}.\\
\textbf{Influência do RaaS.} A oferta de kits e infraestrutura em \textit{RaaS} amplia a frequência e diversidade de ataques, elevando a pressão por detecção proativa \cite{Alzahrani_2025,SACCONE2025104264}.\\
\textbf{Déficit de dados padronizados.} A falta de \textit{datasets} públicos e atualizados limita comparabilidade e reprodutibilidade; muitos estudos operam com bases próprias e pouco documentadas \cite{shaikh2024,CEN2024110138}.

\subsection{Divergências e Desafios}
\textbf{Delimitação da fase pré-criptografia.} Métodos variam entre janelas fixas após execução, primeira chamada a APIs criptográficas e correlação de chamadas de API com eventos de I/O (IRP); esta última é mais precisa, porém exige sincronização temporal e captura mais custosa \cite{shaikh2024}.\\
\textbf{Análise estática vs. dinâmica.} A análise estática é rápida, mas frágil a ofuscação; a dinâmica é mais robusta a evasão, porém cara computacionalmente e alvo de \textit{anti-debugging} \cite{Begovic_2023,CEN2024110138}.\\
\textbf{Plataforma.} A maioria das abordagens foca Windows, onde a incidência é maior. Pesquisas em dispositivos móveis são menos frequentes e concentram-se em Android, deixando iOS pouco explorado \cite{shaikh2024}.\\
\textbf{Velocidade de propagação.} Estudos que analisam tráfego mostram que a rapidez da movimentação lateral afeta a janela útil para intervenção; medir padrões de propagação ajuda a definir limiares e alarmes \cite{alibis2024measuring}.

\subsection{Indicadores e Sinais Operacionais}
Modelos recentes combinam múltiplas fontes: chamadas de API de criptografia, padrões de I/O, criação de processos e alterações em chaves de registro, buscando reduzir falsos positivos. Abordagens baseadas em grafos temporais de correlação de eventos têm se mostrado promissoras para capturar sequências curtas de pré-impacto \cite{rollere2025algorithmicsegmentationbehavioralprofiling}. A integração desses sinais a playbooks de resposta rápida é apontada como caminho prático para conter ataques antes da cifragem.

\subsection{Diretrizes Práticas e Lacunas de Pesquisa}
Os estudos convergem em recomendações de implantação: coleta unificada de eventos de sistema e rede, normalização de logs e uso de sensores próximos ao endpoint para minimizar latência de detecção \cite{CEN2024110138}. Para produção, autores sugerem pipelines de ML com atualizações frequentes de modelo e validação cruzada em tráfego real, evitando overfitting em \textit{datasets} sintéticos \cite{shaikh2024}. Em ambientes sensíveis a desempenho, estratégias híbridas são defendidas: primeiro filtros leves (assinaturas e \textit{rules}) e, em seguida, análise comportamental ou grafos apenas nos eventos suspeitos \cite{Begovic_2023,rollere2025algorithmicsegmentationbehavioralprofiling}.

Persistem lacunas claras: ausência de \textit{benchmarks} públicos, escassez de artefatos de ataque para iOS/macOS e pouca avaliação de \textit{concept drift} em cenários corporativos dinâmicos \cite{shaikh2024}. Há também pouca exploração de métricas de tempo de reação; medir o intervalo entre o primeiro sinal de pré-criptografia e a contenção efetiva ajudaria a calibrar alarmes \cite{alibis2024measuring}. Essas lacunas apontam caminhos para futuras pesquisas, especialmente no que tange a datasets abertos e à validação longitudinal de modelos em produção.
