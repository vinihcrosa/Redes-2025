\documentclass{article}
\usepackage[utf8]{inputenc}
\usepackage[alf]{abntex2cite}
\usepackage{geometry}
\usepackage{comment}
\geometry{
    a4paper,
    left=3cm,
    right=2cm,
    top=3cm,
    bottom=2cm
}


\begin{document}




\section{Modus Operandi: O Ciclo de Ataque de Ransomware}

A sequência de etapas da cadeia de ataque de um ransomware segue um padrão recorrente identificado em grandes bases de incidentes \cite{SACCONE2025104264,CEN2024110138,enisa_2023}: há uma fase inicial de reconhecimento e entrega, seguida de instalação e movimentação lateral (propagação interna na rede), impacto (criptação e sabotagem) e extorsão, as quais são abordadas com mais detalhes em seções subsequentes.


\subsection{Reconhecimento e Entrega}

O atacante inicia o processo realizando um mapeamento dos serviços acessíveis, interfaces de VPN e RDP (Protocolo de Desktop Remoto), e aplicações web, além de perfis de usuários suscetíveis a ataques de phishing. Nesta etapa, vetores recorrentes incluem: o envio de campanhas de phishing e \textit{malspam} (spam malicioso) contendo macros ou links que infectam dispositivos \cite{CEN2024110138,shaikh2024}; a exploração de vulnerabilidades de alta severidade já conhecidas e corrigidas em serviços expostos (CVEs N-day) \cite{SACCONE2025104264}; o uso de (anúncios maliciosos) \textit{malvertising} e downloads automáticos sem consentimento do usuário (\textit{drive-by downloads}) por meio de kits de exploit \cite{CEN2024110138}.

\subsection{Instalação, Persistência e Movimentação Lateral}

Após a entrega bem-sucedida do código, o malware estabelece persistência no sistema, frequentemente por meio de tarefas agendadas, serviços ou modificações no registro, e inicia a coleta de credenciais com o objetivo de ampliar o domínio do atacante. Técnicas comuns incluem o uso de contas válidas e captura de senhas e credenciais do sistema para facilitar o acesso lateral, isto é, o acesso a outros dispositivos na rede \cite{SACCONE2025104264}. Na mesma linha, é possível citar a adoção de ferramentas nativas do sistema, como PowerShell, PsExec e WMIC, para reduzir sinais de detecção \cite{CEN2024110138}, bem como a propagação automática através de redes SMB ou domínios Active Directory mal configurados, cuja velocidade pode ser monitorada em padrões de tráfego \cite{alibis2024measuring}. 

Dentre as diferentes organizações criminosas há distintas formas de operação nesta fase. Grupos de perfil mais generalista, como LockBit e ALPHV, tendem a combinar exploração de vulnerabilidades recentes com ataques de força bruta contra RDP (Protocolo de Desktop Remoto) e VPN, enquanto operadores mais especializados preferem trabalhar com credenciais vazadas e ruído mínimo \cite{SACCONE2025104264}.

\subsection{Criptografia}

Antes de iniciar a etapa de cifragem, o ransomware normalmente desativa defesas internas, elimina backups locais e realiza um inventário de arquivos críticos para maximizar o dano \cite{Begovic_2023}. Entre as técnicas observadas, destaca-se a criptografia seletiva ou intermitente, que acelera o impacto ao mesmo tempo em que reduz a produção de rastros forenses \cite{Begovic_2023, CEN2024110138}, tal qual o uso sistemático da técnica \textit{Data Encrypted for Impact}, que consiste em criptografar os dados da vítima, tornando estes inacessíveis até que haja algum tipo de resgate (pagamento) \cite{SACCONE2025104264} e marca a fase final do ciclo técnico de ataque.  

\subsection{Extorsão e Negociação}

Com os dados já cifrados e, em muitos casos, também exfiltrados, o grupo atacante inicia um canal de comunicação controlado, geralmente via portais Tor ou chats privados, para apresentar suas exigências. Normalmente, a chave de descriptografia é fornecida mediante pagamento, acompanhada da promessa de não vazamento das informações comprometidas. A pressão sobre a vítima pode aumentar por meio de ameaças de publicação dos dados ou mesmo da realização de ataques DDoS \cite{enisa_2023}. 
Em operações de \textit{Ransomware-as-a-Service} (RaaS), que funcionam como programas com modelos de ransomware pré-desenvolvidos, essa etapa envolve também a divisão da receita entre os operadores da infraestrutura criminosa e os afiliados responsáveis pela intrusão \cite{Alzahrani_2025}. Ainda que represente o fechamento do ciclo econômico do ataque, essa fase pode se prolongar caso a vítima negocie parcialmente, atrase ou tente resistir, criando margens adicionais para coerção.

\bibliography{refs}
\end{document}