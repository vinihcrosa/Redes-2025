\section{Modus Operandi: O Ciclo de Ataque de Ransomware}

Com base no modelo unificado apresentado por Saccone et al. e nos fluxos identificados por Sudheer (2024) e Cen et al. (2024), o modus operandi pode ser estruturado em um kill chain de 4 estágios:


\subsection{Fase 1 — Reconhecimento e Entrega}

O grupo criminoso identifica:
\begin{itemize}
  \item infraestrutura exposta,
  \item vulnerabilidades,
  \item funcionários suscetíveis a phishing.
\end{itemize}

O ransomware chega ao sistema via:
\begin{itemize}
  \item phishing,
  \item exploit,
  \item malvertising,
  \item mídias removíveis.
  \item etc.
\end{itemize}

\subsection{Fase 2 — Instalação e Movimentação Lateral}

O malware:
\begin{itemize}
  \item cria persistencia (tarefas agendadas, serviços),
  \item explora falhas internas,
  \item coleta credenciais,
  \item identifica servidores críticos.
\end{itemize}

Grupos avançados usam TTPs (MITRE ATT\&CK) padronizadas, como:
\begin{itemize}
  \item T1569 – System Services
  \item T1078 – Valid Accounts
  \item T1003 – Credential Dumping
\end{itemize}
\subsection{Fase 3 — Destruição}

Envolve:
\begin{itemize}
  \item exclusão de backups locais,
  \item desabilitação de antivírus,
  \item criptografia seletiva ou intermitente,
  \item roubo de dados (dupla extorsão).
\end{itemize}

\subsection{Fase 4 — Extorsão}

O atacante exige pagamento e ameaça:
\begin{itemize}
  \item não devolver o acesso,
  \item vazar dados sensíveis,
  \item realizar ataques adicionais (DDoS, tripla extorsão).
\end{itemize}