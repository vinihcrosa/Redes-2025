\section{Modus Operandi: O Ciclo de Ataque de Ransomware}

O \textit{kill chain} de ransomware segue um padrão recorrente identificado em grandes bases de incidentes \cite{SACCONE2025104264,CEN2024110138,enisa_2023}: uma fase inicial de reconhecimento e entrega, seguida de instalação e movimentação lateral, impacto (criptação e sabotagem) e extorsão. A maturidade de Ransomware-as-a-Service permite que afiliados escolham TTPs conforme alvo e pressa operacional \cite{Alzahrani_2025}.

\subsection{Fase 1 — Reconhecimento e Entrega}

O atacante mapeia a superfície exposta (serviços, VPN/RDP, aplicações web) e perfis de usuários suscetíveis a phishing. Vetores típicos incluem:
\begin{itemize}
  \item phishing e \textit{malspam} com macros ou links \cite{CEN2024110138,shaikh2024};
  \item exploração de CVEs em serviços expostos (T1190), muitas vezes N-day de alta severidade \cite{SACCONE2025104264};
  \item malvertising e \textit{drive-by} via kits de exploit \cite{CEN2024110138};
  \item cadeia de suprimentos e atualizações adulteradas \cite{enisa_2023}.
\end{itemize}

\subsection{Fase 2 — Instalação, Persistência e Movimentação Lateral}

O payload estabelece persistência (tarefas agendadas, serviços) e coleta credenciais para ampliar o domínio:
\begin{itemize}
  \item T1078 (contas válidas) e T1003 (credential dumping) para ganhar acesso lateral \cite{SACCONE2025104264};
  \item uso de ferramentas nativas (\textit{living off the land}) para evitar detecção \cite{CEN2024110138};
  \item propagação automática via SMB/AD mal configurados, com velocidade mensurável em tráfego de rede \cite{alibis2024measuring}.
\end{itemize}
Grupos generalistas (LockBit, ALPHV) tendem a combinar exploits recentes com bruteforce de RDP/VPN, enquanto especialistas preferem credenciais vazadas e menor ruído \cite{SACCONE2025104264}.

\subsection{Fase 3 — Impacto: Criptografia, Sabotagem e Exfiltração}

Antes da cifragem, o malware neutraliza defesas, remove backups locais e inventaria arquivos críticos \cite{Begovic_2023}. Técnicas observadas:
\begin{itemize}
  \item criptação seletiva ou intermitente para acelerar impacto e reduzir rastros \cite{Begovic_2023};
  \item exfiltração prévia de dados sensíveis (dupla/tripla extorsão) \cite{enisa_2023};
  \item uso de T1486 (Data Encrypted for Impact) como etapa final do ciclo técnico \cite{SACCONE2025104264}.
\end{itemize}
Modelos comportamentais recentes indicam aceleração e correlação temporal entre varredura interna, picos de I/O e início da cifragem, úteis para detecção pré-encriptação \cite{rollere2025algorithmicsegmentationbehavioralprofiling}.

\subsection{Fase 4 — Extorsão e Negociação}

Com dados cifrados e/ou exfiltrados, o grupo inicia comunicação controlada (portais Tor, chats) e apresenta exigências:
\begin{itemize}
  \item entrega de chave mediante pagamento e promessa de não vazamento;
  \item ameaça de publicar dados e/ou conduzir DDoS como pressão adicional \cite{enisa_2023};
  \item em operações RaaS, divisão de receita entre operador e afiliado \cite{Alzahrani_2025}.
\end{itemize}
Essa etapa fecha o ciclo econômico do ataque, mas pode ser reaberta caso a vítima negocie parcialmente ou atrase, gerando novas alavancas de coerção.
