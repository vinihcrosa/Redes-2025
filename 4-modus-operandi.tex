\section{Modus Operandi: O Ciclo de Ataque de Ransomware}

A literatura recente descreve um \textit{blueprint} recorrente para campanhas de ransomware: reconhecimento e entrega, instalação com persistência e movimentação lateral, impacto (criptação e sabotagem) e extorsão final. Grandes levantamentos de incidentes mostram que essas fases se repetem com variações táticas conforme a superfície exposta e o perfil da vítima \cite{SACCONE2025104264,CEN2024110138,enisa_2023}.


\subsection{Reconhecimento e Entrega}

Os atacantes mapeiam serviços expostos (VPN, RDP, aplicações web) e perfis suscetíveis a \textit{phishing}. Vetores recorrentes incluem campanhas de \textit{malspam} com macros ou links maliciosos \cite{CEN2024110138,shaikh2024}, exploração de CVEs críticas já conhecidas (N-day) em portas públicas \cite{SACCONE2025104264} e \textit{malvertising}/\textit{drive-by downloads} por kits de exploit \cite{CEN2024110138}. Em alvos corporativos, ataques de força bruta a RDP e coleta de credenciais vazadas complementam a fase inicial \cite{SACCONE2025104264}.

\subsection{Instalação, Persistência e Movimentação Lateral}

Depois da entrega, o malware cria persistência (tarefas agendadas, serviços ou registro) e coleta credenciais para ampliar acesso. Técnicas de movimentação lateral incluem uso de contas válidas, ferramentas nativas como PowerShell, PsExec e WMIC (\textit{living off the land}) e propagação automática via SMB ou domínios Active Directory expostos \cite{CEN2024110138,SACCONE2025104264}. Padrões de tráfego lateral podem ser medidos para estimar velocidade de propagação \cite{alibis2024measuring}. Grupos generalistas (LockBit, ALPHV) costumam mesclar exploração de CVEs recentes com força bruta em RDP/VPN, enquanto operadores especializados preferem operar com credenciais vazadas e menor ruído \cite{SACCONE2025104264}.

\subsection{Criptografia}

Antes de cifrar, operadores desativam defesas locais, removem \textit{backups} e inventariam arquivos críticos para maximizar impacto \cite{Begovic_2023}. São frequentes a criptografia seletiva ou intermitente (para acelerar execução e reduzir rastros) \cite{Begovic_2023,CEN2024110138} e a sabotagem de recuperação (\texttt{T1490}) combinada à ação \textit{Data Encrypted for Impact} (\texttt{T1486}) \cite{SACCONE2025104264}. Em campanhas de dupla extorsão, a exfiltração ocorre antes ou em paralelo à cifragem para aumentar pressão sobre a vítima \cite{enisa_2023}.

\subsection{Extorsão e Negociação}

Com dados cifrados e, muitas vezes, já exfiltrados, os atacantes abrem canais em Tor ou chats privados para negociar valores e prazos. A pressão aumenta por meio de ameaças de vazamento público, leilões em \textit{leak sites} e, em alguns casos, ataques DDoS adicionais \cite{enisa_2023,SACCONE2025104264}. Em operações \textit{RaaS}, a receita é dividida entre operadores da infraestrutura e afiliados responsáveis pela intrusão \cite{Alzahrani_2025}. Contadores regressivos e scripts automatizados de comunicação ajudam a manter coerção contínua até pagamento ou ruptura da negociação.
