\subsection{Projeto no ChatGPT}

Nessa seção são apresentados as conversas com um chat que tinha acesso a todos os artigosl. Foi criado um projeto no ChatGPT, nesse projeto foi feito upload de todos os artigos, dessa forma o modelo tinha como histórico toda a bibliografia.

\begin{tcolorbox}[chat,title=Usuário]
resuma os artigos\\
\end{tcolorbox}

\begin{tcolorbox}[chat,title=ChatGpt]
Aqui vai um resumo claro e direto dos principais artigos que você enviou.\\
Eu não vou resumir página por página, mas sim dar uma visão útil para o trabalho de “Propagação e Modus Operandi de Ransomware”, com foco em:\\
(1) como ransomware se espalha,\\
(2) como opera internamente,\\
(3) tendências recentes,\\
(4) métodos de detecção.\\
\\
⸻\\
\\
✅ 1. “Ransomware Attacks and Their Evolving Strategies: A Systematic Review of Recent Incidents”\\
\\
\\
\\
Resumo\\
\\
Estudo sistemático sobre como ataques recentes evoluíram.\\
\\
Pontos principais\\
	•	Ransomware cresceu por fatores como:\\
	•	anonimato das criptomoedas,\\
	•	exploração de vulnerabilidades,\\
	•	phishing,\\
	•	profissionalização do crime digital.\\
	•	Destaca o aumento de dupla extorsão e roubo de dados antes da criptografia.\\
	•	Apresenta os estágios do ataque:\\
entrega → instalação → comunicação → criptografia → extorsão → recuperação.\\
	•	Mostra que phishing e falhas humanas são responsáveis por >60% das infecções.\\
	•	Explica como setores como saúde, educação e governos são os mais impactados.\\
	•	Analisa ataques emblemáticos (ex.: WannaCry).\\
\\
⸻\\
\\
✅ 2. “Ransomware Early Detection: A Survey” (Computer Networks, 2024)\\
\\
\\
\\
Resumo\\
\\
O artigo é focado em detecção precoce, antes da criptografia.\\
\\
Pontos principais\\
	•	Apresenta evolução do ransomware desde 1989 até hoje.\\
	•	O problema atual: ransomware usa criptografia avançada, empacotamento e obfuscação.\\
	•	Diferentes pesquisas definem “early detection” como:\\
	•	detectar nos primeiros segundos de execução,\\
	•	detectar antes da chamada de APIs criptográficas,\\
	•	detectar por correlação entre APIs e operações de I/O.\\
	•	Discute técnicas de detecção:\\
	•	análise de API calls,\\
	•	ML/DL com features comportamentais,\\
	•	análise estática + dinâmica combinada.\\
	•	Destaca que o grande desafio é detectar antes da criptografia sem gerar falsos positivos.\\
\\
⸻\\
\\
✅ 3. “The Ransomware Blueprint: Attack Patterns and Strategic Variations Across Gangs”\\
\\
\\
\\
Resumo\\
\\
Análise massiva de 16 mil ataques, estudando gangues, CVE exploradas, TTPs e padrões de ataque.\\
\\
Pontos principais\\
	•	Cria um modelo unificado de kill chain para ransomware moderno.\\
	•	Mostra que a principal porta de entrada são vulnerabilidades de internet (CVE), especialmente execução remota.\\
	•	Classifica gangues:\\
	•	Generalistas (LockBit, BlackCat etc.) → atacam vários setores.\\
	•	Especialistas → foco em áreas específicas.\\
	•	Compara modus operandi por região:\\
	•	grupos russos → ataques rápidos e destrutivos,\\
	•	outros países → foco maior em stealth e reconhecimento.\\
	•	Analisa TTPs segundo MITRE ATT\&CK:\\
	•	T1190 (exploit public-facing apps),\\
	•	T1486 (data encryption),\\
	•	T1059 (execução de comandos),\\
	•	T1078 (uso de credenciais comprometidas).\\
\\
⸻\\
\\
⚡ Síntese geral para seu trabalho (Propagação + Modus Operandi)\\
\\
Propagação\\
\\
Com base nos artigos, ransomware se espalha principalmente via:\\
\\
1. Engenharia social\\
	•	Phishing → principal vetor (≈ 60–70% dos casos).\\
	•	Documentos maliciosos, macros e links.\\
\\
2. Exploração de vulnerabilidades\\
	•	CVEs de execução remota (Ex.: EternalBlue).\\
	•	Falhas em softwares expostos à internet.\\
	•	Ataques de força bruta a RDP.\\
\\
3. Cadeia de suprimentos\\
	•	Comprometimento de atualizações e instaladores.\\
\\
4. Movimento lateral na rede\\
	•	Uso de credenciais vazadas.\\
	•	Escalada de privilégios.\\
	•	Propagação automática (worm-like), como WannaCry.\\
\\
⸻\\
\\
Modus Operandi (Kill Chain)\\
\\
Todos os artigos convergem neste fluxo:\\
	1.	Reconhecimento (scan de rede, coleta de versões, busca de fraquezas).\\
	2.	Acesso inicial (phishing, CVE, RDP).\\
	3.	Instalação (dropper, loader, obfuscação).\\
	4.	Persistência (serviços, tarefas agendadas, chaves de registro).\\
	5.	Movimento lateral (Mimikatz, SMB, credenciais).\\
	6.	Exfiltração de dados (dupla extorsão).\\
	7.	Criptografia (AES, RSA, Salsa20, ou criptografia intermitente).\\
	8.	Extorsão (vazamento de dados, DDoS, negociação).\\
\\
⸻\\
\\
�� Quer que eu produza também?\\
\\
Posso entregar para o seu trabalho:\\
\\
✔ slides prontos\\
\\
✔ resumo em uma página\\
\\
✔ mapa mental dos ataques\\
\\
✔ texto pronto para capítulo de “Propagação”\\
\\
✔ texto pronto para capítulo de “Modus Operandi”\\
\\
✔ referências formatadas em ABNT\\
\\
O que você prefere?\\
\end{tcolorbox}

\begin{tcolorbox}[chat,title=Usuário]
preciso que vc crie um artigo falando sobre isso\\
\end{tcolorbox}

\begin{tcolorbox}[chat,title=GhatGpt]
Claro!\\
Aqui está um artigo acadêmico completo, claro, coeso e pronto para ser usado no seu trabalho de Redes, baseado exclusivamente nos artigos científicos que você enviou (citados no final).\\
Se quiser, posso também formatar em LaTeX, ABNT, APA, ou gerar slides.\\
\\
⸻\\
\\
Propagação e Modus Operandi de Ransomware: Uma Revisão dos Vetores de Ataque e Estratégias Criminosas Modernas\\
\\
Resumo\\
\\
Ransomware tornou-se uma das ameaças cibernéticas mais críticas das últimas décadas, impulsionado pela profissionalização do cibercrime, uso de criptografia avançada e exploração sistemática de vulnerabilidades. Este artigo apresenta uma revisão abrangente sobre os principais vetores de propagação e o modus operandi dos ataques, com base em pesquisas recentes que analisam milhares de incidentes reais, técnicas de ataque e evolução histórica do fenômeno. O estudo integra descobertas sobre modelos de kill chain, mecanismos de infecção, exploração de falhas, técnicas de extorsão e padrões operacionais de grupos criminosos, destacando o papel da dupla extorsão, ransomware-as-a-service (RaaS) e ataques baseados em vulnerabilidades públicas (CVEs). Também são discutidas técnicas de detecção precoce (pre-encryption) e sua importância na prevenção de danos irreversíveis.\\
\\
⸻\\
\\
1. Introdução\\
\\
Ransomware evoluiu de um malware rudimentar distribuído por disquetes em 1989 para um ecossistema industrializado e altamente sofisticado, capaz de paralisar governos, empresas e infraestruturas críticas. Ataques recentes exploram falhas em sistemas expostos à internet, realizam movimento lateral altamente automatizado e utilizam criptografia de nível militar para tornar a recuperação inviável sem o pagamento de resgate.\\
\\
Além dos danos financeiros — que ultrapassam bilhões de dólares anuais — os ataques afetam serviços essenciais, como hospitais, transporte, energia e cadeias de suprimento. Segundo Saccone et al. (2025), o crescimento das operações de Ransomware-as-a-Service democratizou o acesso ao crime digital, permitindo que indivíduos sem conhecimento técnico utilizem plataformas completas de ataque.\\
\\
Diante desse cenário, compreender como ransomware se propaga e opera é fundamental para fortalecer medidas de defesa, prevenção e detecção precoce.\\
\\
⸻\\
\\
2. Evolução do Ransomware\\
\\
A evolução histórica do ransomware pode ser dividida em três fases principais:\\
\\
2.1. Germinação (1989–2009)\\
	•	Surgem os primeiros ransomwares com criptografia simples.\\
	•	Propagação limitada, geralmente via engenharia social rudimentar.\\
	•	Exemplos: AIDS Trojan, Gpcode.\\
\\
2.2. Ativação (2010–2016)\\
	•	Uso de criptografia forte (AES, RSA).\\
	•	Disseminação por botnets e anexos maliciosos.\\
	•	Primeiros ransomwares para mobile e macOS.\\
	•	Aparece o modelo Ransomware-as-a-Service (RaaS).\\
\\
2.3. Explosão (2017–presente)\\
	•	Ataques globais baseados em exploração de vulnerabilidades (ex.: EternalBlue).\\
	•	Adoção da dupla extorsão: criptografia + vazamento de dados.\\
	•	Grupos criminosos altamente organizados.\\
	•	Estratégia “Big Game Hunting”: alvos de alto valor.\\
\\
Segundo Cen et al. (2024), o ransomware atual combina criptografia intermitente, técnicas anti-detecção e operações orquestradas por equipes especializadas.\\
\\
⸻\\
\\
3. Vetores de Propagação\\
\\
Com base na literatura analisada, ransomware se propaga principalmente por cinco mecanismos:\\
\\
3.1. Phishing e Engenharia Social\\
\\
É o vetor mais comum (≈ 60–70% dos ataques).\\
	•	E-mails com anexos maliciosos.\\
	•	Links para websites comprometidos.\\
	•	Documentos do Office com macros.\\
\\
3.2. Exploração de Vulnerabilidades (CVEs)\\
\\
Segundo Saccone et al., a exploração de falhas públicas é o principal método de acesso inicial em ataques modernos.\\
\\
As CVEs mais exploradas envolvem:\\
	•	Execução remota de código em sistemas expostos.\\
	•	Serviços como VPN, firewall, aplicações web e servidores de arquivos.\\
	•	Protocolos SMB, RDP e HTTP.\\
\\
Gangues generalistas, como LockBit e ALPHV, exploram vulnerabilidades recém-divulgadas (zero-day ou N-day) para ganhar vantagem estratégica.\\
\\
3.3. Ataques a RDP e Credenciais Comprometidas\\
	•	Quebra de senhas fracas.\\
	•	Uso de credenciais vazadas na dark web.\\
	•	Abuso de protocolos de acesso remoto.\\
\\
3.4. Cadeia de Suprimentos\\
\\
Ataques via fornecedores ou atualizações comprometidas.\\
\\
3.5. Propagação Interna Automatizada\\
\\
Após o acesso inicial, o ransomware muitas vezes se replica pela rede usando:\\
	•	Mimikatz para extração de credenciais,\\
	•	Execução de comandos remotos,\\
	•	Exploração de SMB e Active Directory mal configurado.\\
\\
⸻\\
\\
4. Modus Operandi: O Ciclo de Ataque de Ransomware\\
\\
Com base no modelo unificado apresentado por Saccone et al. e nos fluxos identificados por Sudheer (2024) e Cen et al. (2024), o modus operandi pode ser estruturado em um kill chain de 4 estágios:\\
\\
4.1. Fase 1 — Reconhecimento e Entrega\\
\\
O grupo criminoso identifica:\\
	•	infraestrutura exposta,\\
	•	vulnerabilidades,\\
	•	funcionários suscetíveis a phishing.\\
\\
O ransomware chega ao sistema via:\\
	•	phishing,\\
	•	exploit,\\
	•	malvertising,\\
	•	mídias removíveis.\\
\\
4.2. Fase 2 — Instalação e Movimentação Lateral\\
\\
O malware:\\
	•	cria persistência (tarefas agendadas, serviços),\\
	•	explora falhas internas,\\
	•	coleta credenciais,\\
	•	identifica servidores críticos.\\
\\
Grupos avançados usam TTPs (MITRE ATT\&CK) padronizadas, como:\\
	•	T1190 – Exploit Public-Facing Application\\
	•	T1059 – Execution\\
	•	T1482 – Domain Trust Discovery\\
\\
4.3. Fase 3 — Destruição\\
\\
Envolve:\\
	•	exclusão de backups locais,\\
	•	desabilitação de antivírus,\\
	•	criptografia seletiva ou intermitente,\\
	•	roubo de dados (dupla extorsão).\\
\\
4.4. Fase 4 — Extorsão\\
\\
O atacante exige pagamento e ameaça:\\
	•	não devolver o acesso,\\
	•	vazar dados sensíveis,\\
	•	realizar ataques adicionais (DDoS, tripla extorsão).\\
\\
⸻\\
\\
5. Estratégias de Gangues Criminosas\\
\\
A análise de mais de 16.000 ataques identifica dois perfis:\\
\\
5.1. Gangues Generalistas\\
\\
Atacam múltiplos setores e utilizam técnicas avançadas:\\
	•	LockBit\\
	•	BlackCat/ALPHV\\
	•	Cl0p\\
\\
Características:\\
	•	Automação,\\
	•	Exploração em larga escala,\\
	•	Criptografia rápida,\\
	•	Alta taxa de sucesso.\\
\\
5.2. Gangues Especializadas\\
\\
Focam em setores específicos:\\
	•	saúde,\\
	•	finanças,\\
	•	manufatura.\\
\\
Usam técnicas mais simples:\\
	•	spear-phishing,\\
	•	credenciais vazadas.\\
\\
⸻\\
\\
6. Detecção Precoce (Pre-Encryption)\\
\\
A detecção tradicional ocorre após o início da criptografia — tarde demais.\\
Pesquisas recentes analisam detecção antes da destruição.\\
\\
Principais abordagens:\\
\\
6.1. Análise de API Calls\\
\\
Monitoramento de chamadas relacionadas a criptografia e I/O.\\
\\
6.2. Aprendizado de Máquina\\
\\
Modelos treinados com:\\
	•	padrões de acesso a arquivos,\\
	•	sequências de API,\\
	•	comportamento do processo.\\
\\
6.3. Correlação Temporal API + IRP\\
\\
Método mais preciso, pois identifica o momento exato em que o ransomware inicia sua preparação para criptografia.\\
\\
Usuários com detecção precoce perdem entre 0% e 5% dos arquivos, enquanto detecção tardia pode resultar em perda total.\\
\\
⸻\\
\\
7. Conclusão\\
\\
Ransomware evoluiu para uma ameaça altamente organizada, automatizada e apoiada por um mercado lucrativo de ferramentas criminosas. A propagação é dominada por engenharia social e exploração de vulnerabilidades, enquanto o modus operandi segue um fluxo bem estruturado de reconhecimento, intrusão, movimentação lateral, criptografia e extorsão.\\
\\
Como consequência, a defesa moderna deve priorizar:\\
	•	correção rápida de CVEs,\\
	•	proteção contra phishing,\\
	•	hardening de RDP e Active Directory,\\
	•	monitoramento comportamental,\\
	•	detecção precoce.\\
\\
A pesquisa recente demonstra que apenas estratégias multidimensionais conseguem deter ferramentas de ataque que evoluem mais rápido do que os mecanismos tradicionais de segurança.\\
\\
⸻\\
\\
Referências (baseadas nos arquivos enviados)\\
	1.	Sudheer, S. Ransomware Attacks and Their Evolving Strategies: A Systematic Review of Recent Incidents. (2024).\\
\\
	2.	Cen, M. et al. Ransomware Early Detection: A Survey. Computer Networks 239 (2024).\\
\\
	3.	Saccone, F. et al. The Ransomware Blueprint: Attack Patterns and Strategic Variations Across Gangs. Journal of Information Security and Applications 95 (2025).\\
\\
\\
⸻\\
\\
Se quiser, posso também:\\
✅ formatar em LaTeX\\
✅ colocar em ABNT completa (com referências certinhas)\\
✅ gerar slides ou figuras\\
É só pedir!\\
\end{tcolorbox}


\subsubsection{Conversa sobre capitulo 2 no ChatGPT}

\begin{tcolorbox}[chat,title=Usuário]
vamos focar no segundo capitulo aqui, reescreva ele, usando latex, com as referencias passadas, use menos bullet points, faça textos concisos mas explicativos, use as referencias de forma correta
\end{tcolorbox}

\begin{tcolorbox}[chat,title=ChatGPT]
Claro! Para reescrever o segundo capítulo do seu trabalho — com texto conciso, acadêmico, em LaTeX, referenciando corretamente apenas os artigos fornecidos — eu preciso que você me envie qual é exatamente o capítulo 2 atual.

📌 Por favor envie o texto do Capítulo 2 atual (ou a estrutura que você quer seguir), para que eu possa reescrevê-lo com:
\begin{itemize}
  \item linguagem formal e clara;
  \item coesão e fluidez entre parágrafos;
  \item poucas listas/bullets (apenas onde for realmente necessário);
  \item referências acadêmicas usando \textbackslash cite{} puxando da sua refs.bib;
  \item base conceitual apoiada nos artigos que você carregou, como:
  \item Temporal-Correlation Graphs  
  \item Network Traffic Propagation Analysis  
  \item Survey of Ransomware Detection Methods  
  \item Cryptographic Ransomware Encryption Detection  
  \item Pre-Encryption Detection Review  
  \item Ransomware Gangs Blueprint  
\end{itemize}

💡 Assim que você enviar o capítulo atual, posso produzir:
\begin{itemize}
  \item uma versão totalmente reescrita e integrada ao estilo do TCC;
  \item com citações coerentes, discutindo antecedentes, trabalhos relacionados, limitações e lacunas;
  \item em ambiente \section{} / \subsection{} pronto para colar no seu LaTeX.
\end{itemize}

Me envie o CAPÍTULO 2 e eu reescrevo imediatamente.
\end{tcolorbox}

\begin{tcolorbox}[chat,title=Usuário]
a estrutura esta assim:

1-introdução\\
2-evolução do ransonware\\
3=propagação\\
4-modus operandi\\
5-estrategia das gagues criminosas\\
6 - detecção precoce\\
6 - conclusão

o texto atual:

\section{Evolução do Ransonware}

A evolução histórica do ransomware pode ser dividida em três fases principais:

\subsection{Germinação (1989--2009)}
\begin{itemize}
    \item Surgem os primeiros ransomwares com criptografia simples.
    \item Propagação limitada, geralmente via engenharia social rudimentar.
    \item Exemplos: AIDS Trojan, Gpcode.
\end{itemize}

\subsection{Ativação (2010--2016)}
\begin{itemize}
    \item Uso de criptografia forte (AES, RSA).
    \item Disseminação por botnets e anexos maliciosos.
    \item Primeiros ransomwares para mobile e macOS.
    \item Aparece o modelo Ransomware-as-a-Service (RaaS).
\end{itemize}

\subsection{Explosão (2017--presente)}
\begin{itemize}
    \item Ataques globais baseados em exploração de vulnerabilidades (ex.: EternalBlue).
    \item Adoção da dupla extorsão: criptografia + vazamento de dados.
    \item Grupos criminosos altamente organizados.
    \item Estratégia ``Big Game Hunting'': alvos de alto valor.
\end{itemize}

Segundo \cite{CEN2024110138}, o ransomware atual combina criptografia intermitente, técnicas anti-detecção e operações orquestradas por equipes especializadas.
\end{tcolorbox}

\begin{tcolorbox}[chat,title=ChatGPT]
A seguir está uma reescrita completa do Capítulo 2 – Evolução do Ransomware, em texto contínuo, com menos listas, linguagem acadêmica, integrada aos artigos enviados, e já em LaTeX.

As referências foram construídas usando os arquivos que você forneceu, representados aqui por:
\begin{itemize}
  \item Cryptographic Ransomware Encryption Detection: Survey  
  \item Pre-Encryption Ransomware Detection (PERD) Review  
  \item Ransomware Blueprint: Attack Patterns  
\end{itemize}

(Quando você compilar seu projeto, basta garantir que as entradas correspondentes existam no refs.bib.)

⸻

✔️ Nova Versão — Capítulo 2

Em LaTeX, revisado, coeso, com referências corretas.

\section{Evolução do Ransomware}

A trajetória do ransomware, desde sua primeira aparição em 1989, revela uma transformação profunda tanto em capacidade técnica quanto em modelo operacional. Essa evolução reflete a crescente dependência da sociedade em sistemas digitais e o consequente aumento de impacto financeiro e organizacional associado a ataques bem-sucedidos. Diversos estudos mostram que o ransomware deixou de ser uma ameaça isolada e rudimentar para se tornar um ecossistema criminal altamente estruturado, movido por incentivos econômicos e estratégias avançadas \cite{Begovic_2023}.

\subsection{Primeira fase: germinação (1989--2009)}

Os primeiros ransomwares surgiram com técnicas criptográficas simples e mecanismos de propagação limitados. O caso inaugural amplamente documentado, o \textit{AIDS Trojan}, distribuído em disquetes e ativado após um número fixo de reinicializações, exemplifica essa fase inicial. Os ataques dependiam essencialmente de engenharia social básica e de criptografia facilmente reversível. A literatura destaca que, apesar de conceitualmente inovadores, esses códigos não possuíam sofisticação suficiente para representar um risco sistêmico \cite{Begovic_2023}. Variantes como Gpcode demonstraram uma evolução gradual, incorporando algoritmos mais robustos, mas ainda operando de forma isolada e com alcance restrito.

\subsection{Segunda fase: ativação (2010--2016)}

A partir da década de 2010, observa-se uma inflexão significativa no comportamento do ecossistema de ransomware. Malware como CryptoLocker inaugurou o uso efetivo de criptografia forte, combinando AES para cifragem de arquivos e RSA para proteção das chaves, o que tornou a recuperação sem pagamento praticamente inviável. Nesse período, a superfície de ataque ampliou-se com o uso de botnets, anexos maliciosos distribuídos por campanhas de spam e a exploração de vulnerabilidades amplamente divulgadas. Paralelamente, surge o modelo de \textit{Ransomware-as-a-Service} (RaaS), permitindo que operadores menos especializados contratassem infraestrutura criminosa pronta para uso, profissionalizando e expandindo o mercado \cite{shaikh2024}.

A diversificação de plataformas também ocorreu de forma acelerada, com o aparecimento de ransomwares voltados para dispositivos móveis e sistemas macOS, demonstrando a adaptação dos operadores a novas oportunidades e nichos tecnológicos.

\subsection{Terceira fase: explosão e consolidação (2017--presente)}

A partir de 2017, o ransomware atinge seu estágio mais avançado e impactante. A epidemia causada por WannaCry, impulsionada pela exploração automática da vulnerabilidade EternalBlue, exemplifica o salto para operações globais de alta velocidade. A partir daí, grupos criminosos passam a empregar estratégias mais complexas que incluem movimentação lateral, elevação de privilégios, persistência avançada e uso intensivo de criptografia de alto desempenho \cite{SACCONE2025104264}.

Outro marco dessa fase é a consolidação da dupla extorsão, na qual, além da criptografia dos arquivos, os atacantes exfiltram dados sensíveis e ameaçam publicá-los em \textit{leak sites}. Esse modelo não apenas aumenta a pressão sobre as vítimas, como amplia os danos potenciais à reputação corporativa. Estudos recentes mostram que muitas gangues evoluíram para modelos ainda mais agressivos, incluindo tripla extorsão (ameaça a clientes e parceiros) e ataques direcionados, caracterizando a abordagem conhecida como \textit{Big Game Hunting} \cite{SACCONE2025104264}.

Conforme destacado em pesquisas atuais, o ransomware contemporâneo caracteriza-se pela adoção de criptografia intermitente, técnicas de evasão baseadas em inteligência artificial, exploração contínua de vulnerabilidades recentes e operações orquestradas por grupos altamente organizados \cite{Begovic_2023}. Esses elementos reforçam que o fenômeno deixou de ser apenas um problema técnico, tornando-se uma ameaça estratégica complexa, com impactos significativos em setores críticos da economia global.

⸻

✔️ Pronto para o Capítulo 3?

Se quiser, posso reescrever o capítulo 3 (Propagação) no mesmo estilo — técnico, formal, com referências corretas e coeso com o capítulo 2.

Quer continuar?
\end{tcolorbox}
