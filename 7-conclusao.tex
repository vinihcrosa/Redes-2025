\section{Conclusão}

O ransomware consolidou-se como ameaça crítica, sustentada por um ecossistema \textit{Ransomware-as-a-Service} que padroniza ferramentas, acelera variantes e amplia o alcance de afiliados \cite{SACCONE2025104264,Alzahrani_2025}. As campanhas combinam exploração de serviços expostos e credenciais, \textit{phishing} em escala e movimento lateral automatizado, culminando em cifragem e extorsão roteirizada — frequentemente com exfiltração prévia de dados \cite{CEN2024110138,enisa_2023}.

Os estudos indicam que respostas eficazes dependem de estratégias multilayer: gestão ágil de vulnerabilidades e credenciais, proteção contra \textit{phishing}, endurecimento de RDP/AD e monitoramento comportamental orientado à \textit{kill chain}. A detecção pré-criptografia emerge como peça central para interromper o ataque antes da cifragem irreversível \cite{Begovic_2023,shaikh2024,rollere2025algorithmicsegmentationbehavioralprofiling}. O dimensionamento de tráfego lateral também ajuda a reduzir a janela de propagação \cite{alibis2024measuring}.

Como consequência, a defesa moderna deve priorizar:
\begin{itemize}
  \item correção rápida de CVEs expostas e gestão de credenciais;
  \item proteção contra \textit{phishing} e canais de entrega;
  \item \textit{hardening} de RDP e Active Directory;
  \item monitoramento comportamental com sinais de pré-criptografia;
  \item exercícios de resposta que considerem extorsão e exposição pública.
\end{itemize}

A literatura reforça que apenas abordagens multidimensionais e orientadas a dados conseguem acompanhar a velocidade de evolução das operações de ransomware, conectando prevenção, detecção e resposta em ciclos curtos de adaptação \cite{CEN2024110138,SACCONE2025104264}.
