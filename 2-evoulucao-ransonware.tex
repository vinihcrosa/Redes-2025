\section{Evolução do Ransomware}

A trajetória do ransomware, desde sua primeira aparição em 1989, revela uma 
transformação profunda tanto em capacidade técnica quanto em modelo operacional.
 Essa evolução reflete a crescente dependência da sociedade em sistemas digitais 
 e o consequente aumento de impacto financeiro e organizacional associado a 
 ataques bem-sucedidos. Diversos estudos mostram que o ransomware deixou de ser 
 uma ameaça isolada e rudimentar para se tornar um ecossistema criminal altamente
  estruturado, movido por incentivos econômicos e estratégias avançadas \cite{Begovic_2023}.

\subsection{Primeira fase: germinação (1989--2009)}

Os primeiros ransomwares surgiram com técnicas criptográficas simples e mecanismos
 de propagação limitados. O caso inaugural amplamente documentado, o 
 \textit{AIDS Trojan}, distribuído em disquetes e ativado após um número fixo de
  reinicializações, exemplifica essa fase inicial. Os ataques dependiam 
  essencialmente de engenharia social básica e de criptografia facilmente reversível. 
  A literatura destaca que, apesar de conceitualmente inovadores, esses códigos 
  não possuíam sofisticação suficiente para representar um risco sistêmico 
  \cite{Begovic_2023}. Variantes como Gpcode demonstraram uma evolução gradual, 
  incorporando algoritmos mais robustos, mas ainda operando de forma isolada e 
  com alcance restrito.

\subsection{Segunda fase: ativação (2010--2016)}

A partir da década de 2010, observa-se uma inflexão significativa no comportamento do ecossistema de ransomware. Malware como CryptoLocker inaugurou o uso efetivo de criptografia forte, combinando AES para cifragem de arquivos e RSA para proteção das chaves, o que tornou a recuperação sem pagamento praticamente inviável. Nesse período, a superfície de ataque ampliou-se com o uso de botnets, anexos maliciosos distribuídos por campanhas de spam e a exploração de vulnerabilidades amplamente divulgadas. Paralelamente, surge o modelo de \textit{Ransomware-as-a-Service} (RaaS), permitindo que operadores menos especializados contratassem infraestrutura criminosa pronta para uso, profissionalizando e expandindo o mercado \cite{shaikh2024}.

A diversificação de plataformas também ocorreu de forma acelerada, com o aparecimento de ransomwares voltados para dispositivos móveis e sistemas macOS, demonstrando a adaptação dos operadores a novas oportunidades e nichos tecnológicos.

\subsection{Terceira fase: explosão e consolidação (2017--presente)}

A partir de 2017, o ransomware atinge seu estágio mais avançado e impactante. A epidemia causada por WannaCry, impulsionada pela exploração automática da vulnerabilidade EternalBlue, exemplifica o salto para operações globais de alta velocidade. A partir daí, grupos criminosos passam a empregar estratégias mais complexas que incluem movimentação lateral, elevação de privilégios, persistência avançada e uso intensivo de criptografia de alto desempenho \cite{SACCONE2025104264}.

Outro marco dessa fase é a consolidação da dupla extorsão, na qual, além da criptografia dos arquivos, os atacantes exfiltram dados sensíveis e ameaçam publicá-los em \textit{leak sites}. Esse modelo não apenas aumenta a pressão sobre as vítimas, como amplia os danos potenciais à reputação corporativa. Estudos recentes mostram que muitas gangues evoluíram para modelos ainda mais agressivos, incluindo tripla extorsão (ameaça a clientes e parceiros) e ataques direcionados, caracterizando a abordagem conhecida como \textit{Big Game Hunting} \cite{SACCONE2025104264}.

Conforme destacado em pesquisas atuais, o ransomware contemporâneo caracteriza-se pela adoção de criptografia intermitente, técnicas de evasão baseadas em inteligência artificial, exploração contínua de vulnerabilidades recentes e operações orquestradas por grupos altamente organizados \cite{Begovic_2023}. Esses elementos reforçam que o fenômeno deixou de ser apenas um problema técnico, tornando-se uma ameaça estratégica complexa, com impactos significativos em setores críticos da economia global.
