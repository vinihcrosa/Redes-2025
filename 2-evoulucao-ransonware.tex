\section{Evolução do Ransomware}

A trajetória do ransomware, desde 1989, mostra um salto de códigos rudimentares para operações industriais, guiadas por incentivos econômicos e ecossistemas criminosos maduros. A evolução acompanha a expansão da superfície digital e a profissionalização das gangues, com mudanças claras nas técnicas, vetores de entrega e modelos de extorsão \cite{Begovic_2023,enisa_2023}.

\subsection{Primeira fase: germinação (1989--2009)}

Os primeiros ataques (como o \textit{AIDS Trojan}) usavam criptografia fraca e dependiam de engenharia social simples (disquetes e contadores de reinicialização). Apesar de inaugurar o conceito de pagamento por resgate, essa fase carecia de automação e escala, resultando em impacto limitado \cite{Begovic_2023}. Variantes como Gpcode incorporaram algoritmos mais robustos, mas ainda operavam de forma isolada e sem alvo setorial definido.

\subsection{Segunda fase: industrialização inicial (2010--2016)}

Malwares como CryptoLocker consolidaram o uso combinado de AES e RSA, tornando a recuperação sem chave praticamente inviável e demonstrando viabilidade econômica. Campanhas de \textit{spam}, \textit{exploit kits} e botnets ampliaram a superfície de entrega. Surge o modelo de \textit{Ransomware-as-a-Service} (RaaS), que democratiza a operação criminosa e acelera a proliferação de famílias \cite{shaikh2024,Alzahrani_2025}. A partir de 2014, observa-se a diversificação para plataformas móveis e macOS, explorando brechas em ecossistemas ARM e canais laterais de hardware \cite{2976749.2978360}. Essa fase estabelece a ponte entre experimentação técnica e um mercado clandestino mais estruturado.

\subsection{Terceira fase: consolidação, dupla extorsão e especialização (2017--presente)}

O surto do WannaCry (2017) marcou a combinação de exploração automatizada (EternalBlue) com criptografia massiva, inaugurando a era das operações de alta velocidade. A partir daí, as gangues incorporam movimentação lateral acelerada, elevação de privilégios e persistência avançada como padrão \cite{SACCONE2025104264}. Modelos de dupla e tripla extorsão (cifrar, exfiltrar e ameaçar terceiros) tornam-se dominantes, elevando o valor dos resgates e a pressão sobre vítimas. Relatórios recentes mostram que o ransomware compõe parcela relevante do panorama de ameaças globais, com campanhas orientadas a setores críticos e cadeias de suprimento \cite{enisa_2023}. Estudos de tráfego e kill chain evidenciam ataques que alternam exploração de CVEs críticas, contas válidas e propagação interna automatizada \cite{CEN2024110138}.

No estágio atual, observam-se dois eixos principais: (i) diversificação de técnicas para evasão e cifragem (criptografia intermitente, ofuscação e uso de IA) \cite{Begovic_2023}, e (ii) especialização por alvo — gangues generalistas (LockBit, ALPHV) ampliam escala e sofisticação, enquanto grupos "especialistas" preferem phishing e credenciais reaproveitadas em nichos setoriais \cite{SACCONE2025104264}. A evolução recente também reforça a pesquisa de detecção precoce (pré-encriptação), com abordagens baseadas em rede e comportamento \cite{alibis2024measuring,rollere2025algorithmicsegmentationbehavioralprofiling}.
